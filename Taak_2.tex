\documentclass[11pt]{article}

\author{Groep 6:\\
		Niels Desair\\
		Bram Kelchtermans\\
		Dylan Toirkens}
		
\title{\textbf{Studie van meetbare objectieven en ontwerpprincipes van Shneidermann}}

\date{12/10/2016}

\usepackage{graphicx}
\usepackage{parskip}
\usepackage{float}

\begin{document}
	\begin{titlepage}
		
		\newcommand{\HRule}{\rule{\linewidth}{0.5mm}} % Defines a new command for the horizontal lines, change thickness here
		
		\begin{center} % Center everything on the page
			
			\textsc{\LARGE Universiteit Hasselt}\\[1.5cm] % Nme of your university/college
			\textsc{\Large Humane en sociale aspecten van de informatica}\\[0.5cm] % Major heading such as course name
			
			\HRule \\[0.4cm]
			{ \huge \bfseries Studie van meetbare objectieven en ontwerpprincipes van Shneidermann}\\[0.4cm]
			\HRule \\[1.5cm]
			
			\begin{minipage}{0.4\textwidth}
				\begin{flushleft} \large
					\emph{Groep 6:}\\
					Niels \textsc{Desair} \newline
					Bram \textsc{Kelchtermans} \newline
					Dylan \textsc{Toirkens}
				\end{flushleft}
			\end{minipage}
			~
			\begin{minipage}{0.4\textwidth}
				\begin{flushright} \large
					\emph{Datum:}\\
					12 Oktober 2016
					\emph{Academiejaar: } \\
					2016-2017
				\end{flushright}
			\end{minipage}\\[4cm]
			\vspace{40 mm}
			\includegraphics[width=3.0cm]{uhasselt-logo}\\[2.0cm]  
		\end{center}
	\end{titlepage}

\section{Inleiding}
In deze studie willen wij kijken of een computerprogramma dat dagelijks door duizenden mensen gebruikt wordt goed scoort bij de meetbare objectieven en de principes van Shneidermann. Het programma dat we hiervoor gekozen hebben, is Google Chrome. Deze browser is al enkele jaren de populairste en geniet van een zeer groot aantal gebruikers en wij denken dat de User Interface daar wel iets mee te maken zou kunnen hebben.
\newpage

\section{Bespreking Niels}
individuele bespreking meetbare objectieven en principes van Shneiderman door student 1 (twee tot drie bladzijden, 800 à 1200 woorden)
\subsection{Meetbare objectieven}
\subsection{Ontwerpprincipes van Shneidermann}
\newpage

\section{Bespreking Bram}
individuele bespreking meetbare objectieven en principes van Shneiderman door student 2 (twee tot drie bladzijden, 800 à 1200 woorden)
\subsection{Meetbare objectieven}
\subsection{Ontwerpprincipes van Shneidermann}
\newpage

\section{Bespreking Dylan}
individuele bespreking meetbare objectieven en principes van Shneiderman door student 3 (twee tot drie bladzijden, 800 à 1200 woorden)
\subsection{Meetbare objectieven}
\subsection{Ontwerpprincipes van Shneidermann}
\newpage


\section{Conclusie}
gezamenlijke conclusies over de meetbare objectieven en principes van Shneiderman (drie tot vier bladzijden, 1200 à 1600 woorden). De beoordeling gebeurt voornamelijk op basis van de gezamenlijke conclusie, dus zorg dat deze volledig is (voldoende voorbeelden, argumentering, screenshots ...)!
\newpage
\end{document}